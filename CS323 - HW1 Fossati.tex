\documentclass[11pt, oneside]{article}   	% use "amsart" instead of "article" for AMSLaTeX format
\usepackage{geometry}                		% See geometry.pdf to learn the layout options. There are lots.
\geometry{letterpaper}                   		% ... or a4paper or a5paper or ... 
%\geometry{landscape}                		% Activate for rotated page geometry
%\usepackage[parfill]{parskip}    		% Activate to begin paragraphs with an empty line rather than an indent
\usepackage{graphicx}				% Use pdf, png, jpg, or eps§ with pdflatex; use eps in DVI mode
								% TeX will automatically convert eps --> pdf in pdflatex		
\usepackage{amssymb}

%SetFonts

%SetFonts


\title{CS323 Homework Assignment 1}
\author{Yibo Wang}
\date{January 18, 2017}							% Activate to display a given date or no date

\begin{document}
\maketitle
%\section{}
%\subsection{}

Collaboration Statement: This is my own work. I worked on it by myself without consulting anyone else. 

\vspace{5mm}

\textbf{1. Explanations for ComplexCode4:}

First, let's note that this code has two nested for-loops. The first for loop goes from int i = 1 to when int 1 is less than N and increases by a factor of 2. 

% this < symbol is not working ...

\vspace{5mm}

The second for loop goes from int j = 0 until when j is less than some number N and the overall loop increments by a factor of 2 which can be shown by the math equation $2^{n}$. Therefore the overall asymptotic complexity of the code is log n times N. 

\vspace{5mm}

\textbf{2. Explanations for ComplexCode5:}

The first for-loop decreases by 1 so it goes from some number N to 0. The second for-loop goes from 100 to some number N. The last for-loop goes 15 to some number N*2 but increases by 5. The overall asymptotic complexity is therefore $N^{3}$ based on the three for loops. 

\vspace{5mm}

Overall this code is because the two while loops cause the iterator to move in a second order growth instead of in a linear growth. 
\vspace{5mm}

%
\textbf{3. Explanations for ComplexCode17:}

\vspace{5mm}

The solution is  $N^{2}$. Since there is a for-loop the first loop goes from int i when it is zero to when i is less than some number N so this means that asymptotic complexity is at least N.  The next item to consider is that the if-clause which says !x.contains the array index. This cause is going to add an integer to the arraylist from integers 0 to N since the command checks if the ArrayList x contains the element in the arraylist. So the two factors to consider are the first for loop that goes from integers 0 to N and then the second if statement which adds the element in the arraylist. 

\vspace{5mm}


\textbf{4. Explanations for ComplexCode18:}

This solution is Nlogn. This is because the tree's fundamental coding structure looks like a branch going across and multiplying in an exponential way similar to 1 to 2 to 4 to 8 to 16 to etc. So the logn n makes sense because the data structure that is used is a tree however the N in the solution of Nlogn is because of the for loop which goes from int i = 0 when i is less than N. 

\vspace{5mm}

\textbf{5. Explanations for ComplexCode20:}

The reason that this code is log N is because the recursion allows the function to be repeated exponentially by so for example, it will be N/2 then N/4 then N/8 then N/16 so N is being divided by factors of 2. This means that two to some power x will be equal to N so the running type analysis is log N. 


\end{document}  
